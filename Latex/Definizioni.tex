\documentclass[12pt, a4paper]{article}
\usepackage[italian]{babel}
\usepackage{geometry}
\geometry{margin=2cm}
\usepackage{listings}
\usepackage{xcolor}

\definecolor{codegreen}{rgb}{0,0.6,0}
\definecolor{codegray}{rgb}{0.5,0.5,0.5}
\definecolor{codepurple}{rgb}{0.58,0,0.82}
\definecolor{backcolour}{rgb}{0.95,0.95,0.92}

\lstdefinestyle{mystyle}{
    backgroundcolor=\color{backcolour},   
    commentstyle=\color{codegreen},
    keywordstyle=\color{magenta},
    numberstyle=\tiny\color{codegray},
    stringstyle=\color{codepurple},
    basicstyle=\ttfamily\footnotesize,
    breakatwhitespace=false,         
    breaklines=true,                 
    captionpos=b,                    
    keepspaces=true,                 
    numbers=left,                    
    numbersep=5pt,                  
    showspaces=false,                
    showstringspaces=false,
    showtabs=false,                  
    tabsize=2
}

\lstset{style=mystyle}

\title{Definizioni utili}
\date{}
\author{}
\begin{document}
\maketitle 
\section{Cookie}
Un cookie è un piccolo testo, contenente informazioni uniche, che il web invia nell'header HTTP. Il browser del client tiene una lista di cookie e siti web. Quando l'utente ritorna su un sito web, il browser prenderà automaticamente i cookie (sempre che non siano scaduti).
\begin{itemize}
    \item \textbf{Secure}: il cookie può essere inviato solo su connessioni HTTPS (default: no)
    \item \textbf{HttpOnly}: il cookie non è accessibile via JavaScript (default: no)
    \item \textbf{SameSite}: \begin{itemize}
        \item \textbf{Strict}: può essere inviato solo tra lo stesso dominio
        \item \textbf{Lax}: può essere inviato tra domini diversi con stessa origine
        \item \textbf{None}: nessuna restrizione
    \end{itemize}
\end{itemize}
\subsection{Interazione Client-Server}
\begin{enumerate}
    \item Il client fa una richiesta al server
    \item Il server risponde con un cookie nell'header HTTP (Set-Cookie)
    \item Il client per ogni richiesta successiva invia il cookie nell'header HTTP (Cookie) controllandone la validità
\end{enumerate}
\subsection{In PHP}
\begin{lstlisting}[language=PHP]
setcookie("nome","valore",tempo, "path", "domain", secure, httponly); // set
$_COOKIE["chiave"]; // get
\end{lstlisting}
\section{Sessione}
Il controllo di sessione permette di tener traccia dell'utente durante la sua interazione con un sito web.
\subsection{Interazione Client-Server}
\begin{enumerate}
    \item Il client fa una richiesta al server
    \item Il server risponde con un cookie contentente la session ID nell'header HTTP (Set-Cookie)
    \item Il client per ogni richiesta successiva invia il cookie nell'header HTTP (Cookie) e il server lo usa per identificare la sessione e la validità
\end{enumerate}
\subsection{In PHP}
\begin{lstlisting}[language=PHP]
session_start(); // sempre all'inizio del file php
$_SESSION["chiave"] = valore // set
$_SESSION["chiave"] // get
unset($_SESSION["chiave"]) // rilascio della variabile di sessione
session_destroy() // distrugge la sessione
\end{lstlisting}
\section{PWA}
Una PWA (Progressive Web Application) è un tipo di applicazione web che può essere installata su un dispositivo come applicazione standalone. Devono essere servite su una connessione sicura a causa del Service Worker.
\subsection{Application Shell Architecture}
Prevede il caricamento iniziale di una struttura base dell'applicazione web (applicazione shell) che rimane costante e viene popolata dinamicamente con i dati. Usa chiamate asincrone e modifica del DOM.
\subsection{Service Worker}
Script JavaScript che funziona in background, separato dall'applicazione web, che gestisce la cache, le notifiche push e le richieste di rete, permette di fruire dell'applicazione anche offline.
\subsection{Manifest JSON}
File che descrive come la PWA deve essere visualizzata sui dispositivi degli utenti:
\begin{itemize}
    \item \textbf{name} e \textbf{short\_name}: nome e nome breve della PWA
    \item \textbf{icons}: array di oggetti che rappresentano le icone che verranno usate dalla PWA
    \item \textbf{start\_url}: URL da cui iniziare la navigazione
    \item \textbf{display}: specifica come deve essere visualizzata la PWA
    \item \textbf{theme\_color} e \textbf{background\_color}: per personalizzare l'aspetto della PWA
\end{itemize}
\end{document}